\documentclass[12pt]{article}
\usepackage[spanish]{babel}
\usepackage{graphicx}
\usepackage[usenames,dvipsnames]{xcolor}
\usepackage{color}

%codigos: \begin{abstract} \end{abstract} es resumen
%nueva pagina \newpage
%colores predefinidos : white, black, red, green, blue, cyan, magenta, yellow
%\textcolor{blue}{text}
%\color{blue!20!black!30!green}{Prueba} mezcla de colores, conviene manejar solo 2 colores, es mas manejable

\title{{\LARGE \textbf{\color{blue}{MY TWIN}}}}
\author{Kevin M. Calder\'on\\C\'esar E. SanLucas\\Hern\'an R. Ull\'on}
\date{}

\begin{document}
\maketitle

%\begin{figure}[h]
%\centering
%\includegraphics[scale=0.5]{Hao}     % comentario de figuras, parte del logo
%\end{figure}
%\includegraphics[totalheight=1.2in,width=1in]{Hao}

\newpage
\section{Introducci\'on} 

\textbf{MY TWIN} una aplicaci\'on dirigida a toda clase de usuario mediante la cual podr\'as crear tu propio avatar, el cual se convertir\'a en tu asesor personal de tareas. 

Una aplicaci\'on la cual jugar\'a con los estados de \'animo de tu avatar, los cuales estar\'an definidos por las diferentes tareas que cumplir\'a. Esto es que la aplicaci\'on no solo ser\'a para entretenimiento tendr\'a una funcionalidad que buscar\'a ayudar al usuario de una manera diferente, entretenida y personalizada.\\
\subsection{Tareas} 

\hspace{0.2in}*BATER\'IA: El TWIN  nos va a mostrar\'a el estado de la bater\'ia, indicandonos si se encuentra cargada o descargada, mediante mensajes de voz, acompa\~nado de cambios de \'animo de nuestro avatar, esto es, si se encuentra la bater\'ia cargada el avatar se mostrar\'a contento, del mismo modo si ocurre lo contrario, el \'animo del avatar tambi\'en se ver\'a afectado negativamente. 

\hspace{0.2in}*RECORDATORIOS: Nos ayudar\'a a recordar las fechas importates tales como cumplea\~nos o aniversarios, mostr\'andose el avatar en una imagen de acuerdo al evento a recordar, de un modo poco usual y entretenido, haciendo as\'i que sea mas f\'acil de tener en mente todo el d\'ia aquel acontecimiento importante (Ideal si se nos ha olvidado comprar un regalo).
%\begin{figure}[h]
%\centering
%\includegraphics[totalheight=2in,width=2.5in]{CVPicture}\\\vspace{0.1in}  %figuras extras para parte explicativa
%\includegraphics[totalheight=2in,width=2.5in]{Hao}
%\end{figure}

\end{document}